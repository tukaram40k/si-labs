\section{Appendix: Source Code Listings}

The complete source code for this project is available on my GitHub: \\
\url{https://github.com/tukaram40k/si-labs}.

\subsection{Main Application Code}

\begin{lstlisting}[language=C++, caption={Main application code}]
#include <Arduino.h>
#include <stdio.h>
#include <string.h>
#include "../lib/LedDriver/LedController.h"
#include "../lib/IO/IO.h"

// create a led controller for pin 7
LedController led(7);

void setup() {
  // init the led controller
  led.setup();

  // redirect stdio to serial
  IO::setup();
  printf("Ready\n");
}

void loop() {
  char input[20];

  // check for data on serial port
  if (Serial.available() > 0) {
    scanf("%19s", input);

    // compare the command
    if (strcmp(input, "LED") == 0 or strcmp(input, "led") == 0) {
      char input2[10];
      scanf("%9s", input2);

      if (strcmp(input2, "ON") == 0 or strcmp(input2, "on") == 0) {
        led.turnOn();
        printf("LED is ON\n");
      } else if (strcmp(input2, "OFF") == 0 or strcmp(input2, "off") == 0) {
        led.turnOff();
        printf("LED is OFF\n");
      } else {
        printf("Unknown command\n");
      }
    } else {
      printf("Unknown command\n");
    }
  }
}
\end{lstlisting}

\subsection{LED Controller Header File}

\begin{lstlisting}[language=C++, caption={LED Controller header file}]
#ifndef LED_CONTROLLER_H
#define LED_CONTROLLER_H

#include <Arduino.h>

// class to control led
class LedController {
public:
  // make a new Led Controller on a pin
  LedController(int pin);

  // setup the led pin as output
  void setup();

  // turn led on
  void turnOn();

  // turn led on
  void turnOff();

  // toggle led state
  void toggle();

  // check if led is on
  bool isOn() const;

private:
  int m_pin; // led pin
  bool m_isOn; // led state
};

#endif
\end{lstlisting}

\subsection{LED Controller Implementation}

\begin{lstlisting}[language=C++, caption={LED Controller implementation}]
#include "LedController.h"

LedController::LedController(int pin) : m_pin(pin), m_isOn(false) {

}

void LedController::setup() {
  pinMode(m_pin, OUTPUT);
  digitalWrite(m_pin, LOW);
  m_isOn = false;
}

void LedController::turnOn() {
  digitalWrite(m_pin, HIGH);
  m_isOn = true;
}

void LedController::turnOff() {
  digitalWrite(m_pin, LOW);
  m_isOn = false;
}

void LedController::toggle() {
  if (m_isOn) {
    turnOff();
  } else {
    turnOn();
  }
}

bool LedController::isOn() const {
  return m_isOn;
}
\end{lstlisting}

\subsection{IO Library Header File}

\begin{lstlisting}[language=C++, caption={IO Library header file}]
#ifndef IO_H
#define IO_H

#include <Arduino.h>
#include <stdio.h>

namespace IO {
  // helper for std output
  int serial_putchar(char c, FILE *stream);

  // helper for std input
  int serial_getchar(FILE *stream);

  // setup stream redirection
  void setup();
}

#endif
\end{lstlisting}

\subsection{IO Library Implementation}

\begin{lstlisting}[language=C++, caption={IO Library implementation}]
#include "IO.h"

static FILE serial_stream;

namespace IO {
  int serial_putchar(char c, FILE *stream) {
    if (c == '\n') {
      Serial.write('\r');
    }
    Serial.write(c);
    return 0;
  }

  int serial_getchar(FILE *stream) {
    while (!Serial.available());
    return Serial.read();
  }

  void setup() {
    // start serial
    Serial.begin(9600);

    // redirect std to serial
    fdev_setup_stream(&serial_stream, serial_putchar, serial_getchar, _FDEV_SETUP_RW);
    stdout = &serial_stream;
    stdin = &serial_stream;
  }
}
\end{lstlisting}